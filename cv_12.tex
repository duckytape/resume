%%%%%%%%%%%%%%%%%%%%%%%%%%%%%%%%%%%%%%%
% Deedy CV/Resume
% XeLaTeX Template
% Version 1.0 (5/5/2014)
%
% This template has been downloaded from:
% http://www.LaTeXTemplates.com
%
% Original author:
% Debarghya Das (http://www.debarghyadas.com)
% With extensive modifications by:
% Vel (vel@latextemplates.com)
%
% License:
% CC BY-NC-SA 3.0 (http://creativecommons.org/licenses/by-nc-sa/3.0/)
%
% Important notes:
% This template needs to be compiled with XeLaTeX.
%
%%%%%%%%%%%%%%%%%%%%%%%%%%%%%%%%%%%%%%

\documentclass[a4paper]{deedy-resume-twopage} % Use US Letter paper (letterpaper), change to a4paper for A4

\begin{document}

%----------------------------------------------------------------------------------------
%	TITLE SECTION
%----------------------------------------------------------------------------------------

\lastupdated % Print the Last Updated text at the top right

\namesection{"Ducky"}{Eloise Macdonald-Meyer}{ % Your name
Developer, Learner \\ % Your website, LinkedIn profile or other web address
Enthusiastic Person % Your contact information
}

%----------------------------------------------------------------------------------------
%	LEFT COLUMN
%----------------------------------------------------------------------------------------

\begin{minipage}[t]{0.66\textwidth} % The left column takes up 33% of the text width of the page


  %------------------------------------------------
  % Experience
  %------------------------------------------------

  \section{Experience}

  \runsubsection{ZKM Karlsruhe}
  \descript{Creative Technologist}

  \location{October 2017 – Present | Karlsruhe, DE}
  \vspace{\topsep} % Hacky fix for awkward extra vertical space
  \begin{tightitemize}
  \item Researched and developed technology for exhibition spaces.
  \item Implemented and advised on works for the Open Codes Exhibition
  \item Created documentation and support for Hackathon and Exhibit in India
  \end{tightitemize}
  Main tech used: \textbullet{} Python \textbullet{} HTML/CSS/JS \textbullet{} Alexa \textbullet{} HTC Vive \textbullet{} SmartPen \textbullet{}

  \sectionspace % Some whitespace after the section

  %------------------------------------------------

  \runsubsection{Takeflight}
  \descript{Web Applications Engineer}

  \location{March 2015 – July 2017 | Hobart, AUS}
  \begin{tightitemize}
  \item Worked in a small team in both backend and frontend.
  \item Documentation of internal tools \& client support documentation.
  \item Promoted takeflight and it's work through conferences and talks on tools.
  \end{tightitemize}
  Main tech used: \textbullet{} Django (python) \textbullet{} Wagtail (CMS) \textbullet{} Node.js \textbullet{} scss \& jinja

  \sectionspace % Some whitespace after the section

  %------------------------------------------------

  \runsubsection{University of Tasmania}
  \descript{Telephony Technician}

  \location{May 2014 – January 2015 | Hobart, AUS}
  \begin{tightitemize}
  \item Provided Telephony support across all UTAS campuses, including interstate.
  %\item Support included, but was not limited to: Repairing and refitting DECT and desk phones \textbullet{} Node Room Maintenance \textbullet{} Client liason \& customer service \textbullet{} Maintenance of telephony equipment \\
  \item Inventory Management of equipment \& Database Management.
  %\item Database Management of internal telephony databases %including tracking of billing and management of address books.\\
  \item Programmed and deployed phones in the UTAS UNIX based systems.
  \end{tightitemize}
  Main tech used: \textbullet{} Unix

  \sectionspace % Some whitespace after the section


  %------------------------------------------------
  % Projects
  %------------------------------------------------

  \section{Projects}

  \runsubsection{SpaceRater}
  \descript{Designer, Web Developer, and AR expert}

  \location{March 2018 | Karlsruhe, DE}
  Designed and partly implemented at the Open Codes Hackathon, this was a mobile AR experience for visualising peoples movements and feelings through 3D space.
  %Using Wifi Access Points and a users smartphones to visualising peoples movement and feelings through 3D space
  %application used a range of tracking and experience data to visualise peoples journey and experiences in 3D space.
  %This was all  accessible in a mobile browser for the user, and a content management system for the space owner.
  %This application would track users mobiles in 3D space as they moved through the exhibition, as well as allow users to rate different pieces.
  %Users could then see where they and others had been through a visualistion of a coloured line floating in 3D space.
  %People's experiences would be shown by a colour change in the line, using the ratings they had given.
  %I put together a web application with a content management system to allow space owners to change ratings, as well as the AR views.

  \sectionspace % Some whitespace after the section

  %------------------------------------------------
  \runsubsection{Binnable}
  \descript{Captain}

  \location{Nov 2016 | Santa Cruz, CA}
  After pitching a mobile application that used computer vision to educate and gamify recycling. % Binnable (then unnamed) was selected as one of the top 10 ideas for the weekend.
  I then lead the team in researching and prototyping the application.
  Including making and taking surveys around the town. % At the end of 48 hours a polished pitch was delivered.

  \sectionspace % Some whitespace after the section

  %------------------------------------------------

  \runsubsection{Puzzle Cube}
  \descript{Game Designer and Technical Writer}

  \location{February 2016 | Adelaide, AUS}
  %For the end of the augmented reality summer school a final project was made in a team of three. The puzzle cube was a physical cube with markers on each side and was originally designed to have escher like puzzles augmented inside the cube. However by the delivery of this project it had extended to have many different games augmented onto the cube.
  The final project for AR summer school. Using a physical cube to augment games "inside" and "outside" of it. The box quickly expanded beyond only escher based puzzles.%why does this return
  %The whole team was involved in laser printing the cubes, and setting up the project using ARToolKit and Unity3D.

  \sectionspace % Some whitespace after the section

  %------------------------------------------------

  \runsubsection{AirCondor}
  \descript{Frontend Developer and Map Wrangler}

  \location{July 2015 | Hobart, AUS}
  An air conditioner recommendation app. Using processed air conditioner cooling data and energy ratings with an easy to use, and queryable frontend on it.

  \sectionspace % Some whitespace after the section

  %------------------------------------------------

  \runsubsection{What is Gov?}
  \descript{Data Wrangler}

  \location{July 2014 | Hobart, AUS}
  An iOS game designed to educate players about agencies and roles of the Australian Government, using data on Commonwealth Government Agencies.

  \sectionspace % Some whitespace after the section

  %------------------------------------------------

  \runsubsection{Marvellous Ultimate Appliance}
  \descript{Visual \& Game Designer}

  \location{May 2013 | Hobart, AUS}
  An electronic card game designed to help raise awareness of the energy consumed by common household appliances. Energy data is used to provide stats for cards.%representing each appliance in the E3 Program database.
  \sectionspace % Some whitespace after the section

  %------------------------------------------------
  % Awards
  %------------------------------------------------

  %\section{Awards}

  %\begin{tabular}{rll}
  %2016	 & APAC & Google WTM/Anita Borg Memorial Scholar\\
  %2016	 & Australia & Start Up Catalyst Youth Mission Recipient \\
  %2016	 & Worldwide & Xbox Game Changer for GDC'17\\
  %2015	 & Tasmania & Most Commercially Viable Project, Govhack\\
  %2014	 & Aus and NZ & Govhack Best Use of National Archives of Australia Data \\
  %2014	 & Aus and NZ & Govhack Best Open Government Data Hack \\
  %2014	 & Aus and NZ & Govhack 2\textsuperscript{nd} Best Digital Humanities Hack \\
  %2013	 & Aus and NZ & Govhack Best Use of data.gov.au \\
  %\end{tabular}

  %\sectionspace % Some whitespace after the section

  %------------------------------------------------
  % Presentations
  %------------------------------------------------

  %\section{Presentations}

  %\begin{tabular}{rll}
  %2017 & Nz.js conf & Augmenting Reality with JavaScript\\
  %2016 & OSCONeu & I just want to talk about wagtail and how great it is\\
  %2016 & BuzzConf & ARgh: Not so real monsters!\\
  %2016 & Pyconau & Mental Health in Development\\
  %2016 & PAXaus & Panel: My Friends keep leaving and it's ruining boardgames day\\
  %\end{tabular}

  %\sectionspace % Some whitespace after the section

%----------------------------------------------------------------------------------------

\end{minipage} % The end of the left column
\hfill
%
%----------------------------------------------------------------------------------------
%	RIGHT COLUMN
%----------------------------------------------------------------------------------------
%
\begin{minipage}[t]{0.33\textwidth} % The right column takes up 66% of the text width of the page

  %------------------------------------------------
  % Education
  %------------------------------------------------

  \section{Education}

  \subsection{RMIT}

  \descript{B.Tech}
  \location{May 2018 | Melbourne,AUS}
  \location{Previously B.Sc \& B.Comp UTAS}
  Previous Majors in Games Technology and Geographic Information Systems \\
  %President of Computing Society %\textbullet{}
  %Marketing and Sponsorship Manager for UTAS Robogals \textbullet{}
  %Presenter at Edge Radio \textbullet{}
  %Treasurer of TAMS \textbullet{}
  %Delegate and Guide for White Water Rafting Club \textbullet{}
  %PC and Handheld Coordinator for DigiVisiPop

  \sectionspace % Some whitespace after the section

  %------------------------------------------------
  \hyphenpenalty 30000
  \subsection{University\hbox{ }of South~Australia}

  \descript{AR Summer School}
  \hyphenpenalty 50

  \location{Summer 2016 | Adelaide,AUS}
  Augmented Reality Summer school, co-run by DAQRI. \\
  %Introduction to research and techniques in AR using ARtoolkit.
  %Started Escher/Puzzle Cube Project \\

  \sectionspace % Some whitespace after the section

  %------------------------------------------------

  \subsection{The Friends' School}

  \location{Grad. Dec 2010 | Hobart,AUS}
  English, Sciences(Chemistry and Biology), Philosophy, Maths, Japanese \\
  %Distinctions in National Maths and Science challenges \\
  %1st Women's Badminton, Hockey and Orienteering Teams \\
  %Marimba Ensemble, 2nd in National Percussion Eisteddfod \\
  %Jazz Band and Orchestra, Percussion and Clarinet \\
  %Sound and Lightning Technician for Drama, Ceramic Club \\

  \sectionspace % Some whitespace after the section

  %------------------------------------------------
  % Links/Orgas and volunteering
  %------------------------------------------------

  %\section{Volunteering}

  %\subsection{Organisational Role}
  %\begin{tabular}{ll}
  %Tasjam & Founder \\
  %TUCS & President \\
  %linux.conf.au 2017 & Organiser \\
  %CompCon & Co-Founder \\
  %Startup Tas & Committee \\
  %ACS & Committee/YiTtas Chair \\
  %Girl Geek Coffees & Ambassador \\
  %Robogals & Sponsorship and marketing manager \\
  %Django Girls & Workshop Organiser \\
  %Entropia & Etching Supervisor \\
  %\end{tabular}
  %\subsection{Volunteer}
  %PAXaus \textbullet{} Surprise Attack \textbullet{} GCAP \textbullet{} Unite Melbourne \textbullet{}
  %34C3 Angel \textbullet{} pyconau \textbullet{} kiwipycon \textbullet{} BuzzConf \textbullet{} Amnesty International
  %\textbullet{} Cancer Council Australia \textbullet{} Guide Dogs Association \\


  %\section{Links}

  %Github:// \href{https://github.com/duckytape}{\bf duckytape} \\
  %Gitlab:// \href(https://gitlab.com/duckytape} {\bf duckytape} \\
  %LinkedIn:// \href{https://www.linkedin.com/in/eloiseducky}{\bf eloiseducky} \\
  %Twitter:// \href{https://twitter.com/ducky_tape}{\bf @ducky_tape} \\
  %Gitbook:// \href{https:duckytape.gitbooks.io//}{\bf duckytape} \\

  %\sectionspace % Some whitespace after the section

  %------------------------------------------------
  % Coursework / Presentations
  %------------------------------------------------

  %\section{Presentations}

  %\subsection{Talks}
  %\location{2017 | Nz.js conf}
  %Augmenting Reality with JavaScript \\
  %\location{2016 | OSCONeu}
  %I just want to talk about wagtail... \\
  %\location{2016 | BuzzConf}
  %ARgh: Not so real monsters! \\
  %\location{2016 | Pyconau}
  %Mental Health in Development \\
  %\subsection{Panels}
  %\location{2016 | PAXaus}
  %My Friends keep leaving and it's ruining boardgames day \\
  %\subsection{Workshops}
  %\location{2016 | National Science Week}
  %Web Women Weekend (Django and Rails) \\
  %\location{2014-Present | Python in Schools}
  %Coding for kids and families

  %\sectionspace % Some whitespace after the section

  %------------------------------------------------
  % Skills
  %------------------------------------------------
  \sectionspace
  \section{Skills}

  \subsection{Programming}
  \location{Most used:}
  \textbullet{} C \textbullet{} Python {\footnotesize \textit{\textbf{(esp. Django \& WagtailCMS)}}} \\
  \textbullet{} HTML/Jinja \textbullet{} CSS/SCSS \textbullet{} \LaTeX\ \\
   \textbullet{} JavaScript {\footnotesize \textit{\textbf{(esp. AR.js, tracking.js, Aframe)}}} \\

  \sectionspace

  \location{Experimenting:}
  \textbullet{} C++ \textbullet{} C\# \textbullet{} ARKit \textbullet{} ARToolKit \\
  \textbullet{} Swift \textbullet{} Ruby {\footnotesize \textit{\textbf{(esp. Rails \& Sinatra)}}} \\

  \sectionspace

  \location{Tools:}
  \textbullet{} Unity3D \textbullet{} Godot \textbullet{} Emacs \\
  \textbullet{} Xcode \textbullet{} Android studio \textbullet{} vim \\
  \textbullet{} ArcGIS/QGIS \textbullet{} Wordpress \\
  \textbullet{} Amazon Alexa Flash Skills\\

  \sectionspace

  \subsection{Other}
  Game Design \textbullet{} Team Work \\
  \textbullet{} Enthusiasm \textbullet{} Customer Service \\
  \textbullet{} Sailing \textbullet{} Manual Drivers License\\

  \sectionspace

  \location{Languages:}
  English {\footnotesize \textit{\textbf{(Native)}}} \textbullet{} German {\footnotesize \textit{\textbf{(A2)}}} \\
  \textbullet{} Japanese {\footnotesize \textit{\textbf{(Conversational)}}} \\

  \sectionspace % Some whitespace after the section

  \sectionspace
  \section{Links}

  Github:// \href{https://github.com/duckytape}{\bf duckytape} \\
  Gitlab:// \href{https://gitlab.com/duckytape} {\bf duckytape} \\
  LinkedIn:// \href{https://www.linkedin.com/in/eloiseducky}{\bf eloiseducky} \\
  Twitter:// \href{https://twitter.com/ducky_tape}{\bf ducky\_tape}\\
  Gitbook:// \href{https:duckytape.gitbooks.io//}{\bf duckytape} \\

  \sectionspace % Some whitespace after the section


%----------------------------------------------------------------------------------------

\end{minipage} % The end of the right column

%----------------------------------------------------------------------------------------
%	SECOND PAGE (EXAMPLE)
%----------------------------------------------------------------------------------------

\newpage % Start a new page

\begin{minipage}[t]{0.66\textwidth} % The left column takes up 33% of the text width of the page


  \section{Presentations}

  \runsubsection{Augmenting Reality with JavaScript}
  \descript{NZ.js Conf}

  \location{2017, New Zealand}
  Augmented Reality (AR) is becoming more accessible to a wide range of users, and with it, it brings a unique way for applications to interact with reality. So how can JavaScript be used to create these experiences?
  \sectionspace % Some whitespace after the section

  %------------------------------------------------

  \runsubsection{I just want to talk about Wagtail...}
  \descript{OSCONeu}

  \location{2016, United Kingdom}
  An introduction to Wagtail, an open source content management system built on Django; the pros and cons of the system, its unique features (like streamfields), and  how to get started,
  %and the future of the project. Leading to a short demonstration of quickly and easily building a site with Wagtail.

  \sectionspace % Some whitespace after the section

  %------------------------------------------------

  \runsubsection{ARgh: Not so Real Monsters}
  \descript{BuzzConf}

  \location{2016, Australia}
  This talk will briefly introduce Augmented Reality, compare some of development kits available for it and how to get started with them.
  Throughout, It will also discuss some of the ways AR can be implemented more meaningfully and the future!
  \sectionspace % Some whitespace after the section

  %------------------------------------------------

  \runsubsection{Mental Health in Development}
  \descript{Pyconau}

  \location{2016, Australia}
  This talk will explore the various mental health pressures and impacts that development can have, as well as the strategies developers, and their workplaces, can use to deal with them to support those who may be suffering.

  \sectionspace % Some whitespace after the section

  %------------------------------------------------

  \runsubsection{My friends keep leaving and it's ruining games day}
  \descript{PAXaus (Panel)}

  \location{2016, Australia}
  A conversation about what you can do to keep playing games with mates despite distances holding us apart.

  \sectionspace % Some whitespace after the section

  %------------------------------------------------
  %\sectionspace
  \section{Workshops}

  \runsubsection{Games \& FOSS Miniconf}
  \descript{Linux.conf.au}

  \location{2017 \& 18, Hobart \& Sydney, AUS}
  \vspace{\topsep} % Hacky fix for awkward extra vertical space
  \begin{tightitemize}
  \item Worked with one other organiser to promote and schedule the conf.
  \item Requested and reviewed presentations, and further speaker liason %while working with accepted speakers to polish presentations and make sure they could logistically attend.
  \item Prepared gifts for speakers (Hama Bead penguins 2016 and Penguin shaped PCB's with LEDs 2017).
  \item Researched and collected examples of FOSS in games %for a play session through communications with Game Developers and Companies.
  \end{tightitemize}
  Main tech used: \textbullet{} AR.js \textbullet{} Unity3D \textbullet{} ScummVM \textbullet{} Deckset

  \sectionspace % Some whitespace after the section

  %------------------------------------------------

  \runsubsection{Web Women Weekend}
  \descript{National Science Week}

  \location{2016, Hobart, AUS}
  \vspace{\topsep} % Hacky fix for awkward extra vertical space
  \begin{tightitemize}
  \item Promoted the event for participants to attend and invited mentors.
  \item Mentored and guided participants through ice breaker sessions and the Django Girls and Rails Girls tutorials
  \item Mentoring including trouble shooting of participants code and helping lead participants to answers and understanding
  \end{tightitemize}
  Main tech used: \textbullet{} Django (python) \textbullet{} Rails (ruby) \textbullet{} html \textbullet{} css

  \sectionspace % Some whitespace after the section

  %------------------------------------------------

  \runsubsection{Beginner Coding Workshops}
  \descript{Python in Schools}

  \location{2014 - Ongoing | Tasmania, Australia}
  \begin{tightitemize}
  \item Created online resources for teaching children about computers and coding on Gitbook.
  \item Created modular sprite based game in python using pygame. %This was to be used and modified by participants and was thus well documented readable.
  %\item Hosted workshops with schools and other family oriented organisations. \\
  \item Recruited and trained volunteers to be mentors and help in these workshops.
  \item Contact and maintained relationships with schools, universities and other educational organisations.
  %\item Designed, researched and built workshop to introduce core programming concepts and the command line to students and teachers.
  %\item Created simple interactive exercises in python to teach students concepts that they could then use in a game.
  %\item Built modular game in Python for students to use learnt concepts to modify.
  %\item Formatted course materials into a gitbook.
  %\item delivered workshop throughout the state at schools, universities and other learning centres, by recruiting and training volunteer mentors in the materials
  \end{tightitemize}
  Main tech used: \textbullet{} Gitbook \textbullet{} Python \textbullet{} Pygame \textbullet{} HTML/CSS/JS

  \sectionspace % Some whitespace after the section

  %------------------------------------------------
  % Awards
  %------------------------------------------------
  %\sectionspace
  %\section{Awards}

  %\begin{tabular}{rll}
  %2016	 & APAC & Google WTM/Anita Borg Memorial Scholar\\
  %2016	 & Australia & Start Up Catalyst Youth Mission Recipient \\
  %2016	 & Worldwide & Xbox Game Changer for GDC'17\\
  %2015	 & Tasmania & Most Commercially Viable Project, Govhack\\
  %2014	 & Aus and NZ & Govhack Best Use of National Archives of Australia Data \\
  %2014	 & Aus and NZ & Govhack Best Open Government Data Hack \\
  %2014	 & Aus and NZ & Govhack 2\textsuperscript{nd} Best Digital Humanities Hack \\
  %2013	 & Aus and NZ & Govhack Best Use of data.gov.au \\
  %\end{tabular}

  %\sectionspace % Some whitespace after the section


\end{minipage} % The end of the left column
\hfill
\begin{minipage}[t]{0.33\textwidth} % The right column takes up 66% of the text width of the page

  %\section{Links}

  %Github:// \href{https://github.com/duckytape}{\bf duckytape} \\
  %Gitlab:// \href{https://gitlab.com/duckytape} {\bf duckytape} \\
  %LinkedIn:// \href{https://www.linkedin.com/in/eloiseducky}{\bf eloiseducky} \\
  %Twitter:// \href{https://twitter.com/ducky_tape}{\bf ducky\_tape}\\
  %Gitbook:// \href{https:duckytape.gitbooks.io//}{\bf duckytape} \\

  %\sectionspace % Some whitespace after the section

  %------------------------------------------------
  % Links/Orgas and volunteering
  %------------------------------------------------

  \section{Volunteering}

  \subsection{Organisational Role}
  \begin{tabular}{l p{.35\textwidth}}
  Tasjam & Founder \\
  TUCS & President \\
  linux.conf.au & Conference/ Minconf Organiser \\
  CompCon & Co-Founder \\
  Startup Tas & Committee \\
  Girl Geek Coffees & Ambassador \\
  ACS & Committee/ YiTtas Chair \\
  Robogals & Sponsorship \& Marketing \\
  Django Girls & Workshop Organiser \\
  Entropia & Etching (PCB) Supervisor \\
  \end{tabular}

  \sectionspace
  \subsection{Volunteer}
  \begin{tabular}{p{.9\textwidth}}
  \textbullet{} linux.conf.au \textbullet{} Djangconeu \textbullet{} GCAP\\
  \textbullet{} 34c3 Angel \textbullet{} BuzzConf \textbullet{} kiwipycon \\
  \textbullet{} pyconau \textbullet{} Amnesty International \\
  \textbullet{} A Maze Berlin \textbullet{} Surprise Attack \\
  \textbullet{} Unite MEL \textbullet{} Guide Dogs AUS \\
  \textbullet{} Cancer Council AUS \textbullet{} PAXaus \\
  \end{tabular}

  \sectionspace % some whitespace after the section

  \section{Awards}

  \location{2017}
  \descript{GDAA | Australia }\
  \textbullet{} GDC Assist \\
  \location{2016}
  \descript{ Google | APAC }
  \textbullet{}WTM/Anita Borg Memorial Scholar \\
  \descript{ Start Up Catalyst | Aus}
  \textbullet{}Youth Mission Recipient \\
  \descript{ Xbox | World}
  \textbullet{}Game Changer for GDC'17 \\
  \location{2015}
  \descript{Govhack | Tas}
  \textbullet{} Most Commercially Viable Project\\
  \location{2014}
  \descript{ Govhack | Aus/NZ }
  \textbullet{} Best Use of National Archives of Australia Data \\
  \textbullet{} Best Open Government Data Hack \\
  \textbullet{} 2\textsuperscript{nd} Best Digital Humanities Hack \\
  \location{2013}
  \descript{ Govhack | Aus/NZ }
  \textbullet{} Best Use of data.gov.au \\

  \sectionspace % Some whitespace after the section


  \section{Thank You :)}
  May this little box disrupt the repetitive nature of resume \\
  reading. This box isn't meant \\
  to sway you, but I figure you \\
  probably have to read a few \\
  of these, so thank you for your \\
  time and reading my resume!


\end{minipage} % The end of the right column

%----------------------------------------------------------------------------------------

\end{document}
